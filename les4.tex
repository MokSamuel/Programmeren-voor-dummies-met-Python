\chapter{Les 4: Eindopdracht}

De details van de eindopdracht staan op Blackboard. Inhoudelijk wordt er hier een voorzet gegeven.

Het doel is om een script te schrijven die een bewerking voor je automatiseert. Zo kan je bijvoorbeeld denken aan een simpele UV-VIS meting. Als data heb je een bestand met concentraties, bijbehorende extincties en de extincties van onbekende samples, net zoals in les 3 als voorbeeld is gebruikt. Je kan dan een script maken in Python die een dergelijk excel-bestand inleest, een kalibratiecurve maakt, en met de gegevens de concentraties van de samples berekent inclusies foutmarges. 

Er zijn natuurlijk oneindig veel bewerkingen die je kan doen en die je dus ook kan automatiseren. Maak een keus en schrijf je script. De belangrijkste tip: gebruik google! Zoek (in het Engels) wat je wil weten, en de kans is groot dat iemand het al eens heeft gedaan. Het is niet noodzakelijk dat je iedere regel code zelf hebt bedacht. Het is uiteraard geen enkel probleem om code van anderen over te nemen, als het eindproduct maar iets is wat je persoonlijk hebt gemaakt en kan gebruiken. 
De stof van les 3 (pandas, numpy, matplotlib) zijn essentieel voor het goed uitwerken van deze opdracht.

Het doel van deze opdracht is dus iets automatiseren wat je anders handmatig deed, en niet zozeer om te laten zien dat je een top-programmeur bent. De diepgang van de code zal dus per persoon verschillen: de een zal een ingewikkelder programma schrijven dan de ander. Dat is geen probleem, als je maar laat zien dat je een bewerking kan automatiseren.

Dit is ook de reden dat iedereen een persoonlijke opdracht krijgt. Op Blackboard staat er informatie over hoe dit te werk gaat.

Het eindproduct van deze opdracht is de code die je hebt gemaakt, de input die je gebruikt, en de output die het genereert, en daarbij een  beschrijving en toelichting. Lever het aan als .docx of .pdf bestand in Blackboard. Zorg ervoor dat je ook data mee levert zodat de docent de werking van het programma kan controleren.  